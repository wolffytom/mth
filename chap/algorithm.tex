% Copyright (C) 2019 Cui Jialiang ( SESS, PKU ). All rights reserved.

\chapter{基于CNN和RNN的像素级别跟踪算法}
本章是本文的重点,将详细介绍本研究的理论创新.
\par
本研究所作的重点工作实际上是运用了深度学习技术,结合加密-解码结构的图像分割算法,加入RNN等结构实现了像素级别目标跟踪.具体算法结构,实现细节,训练等将在本章重点介绍.
\section{算法结构与核心思想}
本文所实现的算法将首先基于用于实现静态图像分割的U-Net\supercite{ronneberger2015u}的多级降级-升级卷机神经网络结构.但将在这个多级网络结构中加入Conv-LSTM结构.
\par
类似与U-Net的结构,本文的卷机网络部分也将有多个降级和升级结构;每个降级结构包括几个卷积层,使用池化结构进行降级;每个升级部分采用升卷积进行升级处理.在降级过程中,图片数据的尺寸大小会衰减,同时等比例增加其波段范围.对于3层的结构,最小级的波段将有128个.这个多级结构的设计理念是为了处理多尺度问题;浅层的级别能很好的处理细节问题,但对宏观的把控会较弱,具体表现为可能会出现噪声点;深层的结构对宏观把控好,但对边界处理较弱.升级结构能将浅层处理得到的边界信息与深层处理得到的宏观信息相结合,得到一个更好的结果.
\par
在时间尺度,本研究的算法将主要采用LSTM算法解决问题.具体的,LSTM单元将被加入到各个层级当中.LSTM在各种跟踪算法中有广泛应用,但大多数算法仅仅将其作为对最后结果的处理手段.本研究的算法将把LSTM作为所有的中间状态记录单元.
\par
与纽约大学2017年实现的Conv-LSTM结构的跟踪算法不同的是,本文所采用的多级神经网络将把Conv-LSTM加入各个卷机层级;而与U-Net,SegNet等多级分割算法不同的是,本文将在整个结构中多处穿插LSTM以得到一个时间连续的结果.
\section{基于CNN和RNN的像素级别跟踪模型在空间维度的处理}
\subsection{多尺度思想的引入}
\section{基于CNN和RNN的像素级别跟踪模型在时间维度的处理}
\subsection{跟踪状态}
\subsection{跟踪系统初始化}
% Copyright (C) 2019 Cui Jialiang ( SESS, PKU )。 All rights reserved。

\chapter{致谢}
首先诚挚的感谢我的导师赵红颖老师,赵老师悉心的教导使我得以一窥数字图像处理领域的深奥,不时的讨论并指点我正确的方向,使我在这些年中获益匪浅。赵老师对学问的严谨更是我辈学习的典范。在此,我向我的指导老师表示最诚挚的谢意!本论文的完成另外亦得感谢实验室的冯朝晖,李芹,刘旭林同学和已经毕业的师兄师姐们的协助。因为有你们的帮忙,本论文能够更完整而严谨。
\par
感谢父母在我研究生期间对我的支持。
\par
感谢我的室友李昊远,胡安冬,王雯,几年同寝,无数次开黑,你们的帮忙我铭感在心。感谢在山鹰社遇到的志同道合的朋友们,愿鹰们能继续远走高飞。
\par
本论文在写作过程中使用了Overleaf在线编辑器 \url{https://www。overleaf。com/} ,Visual Studio Code文件编辑器 \url{https://code。visualstudio。com/} ,TexLive \url{https://www。tug。org/texlive/} 等免费软件。本论文使用了Casper Ti。 Vector同学及前辈制作的论文模版 \url{https://gitlab。com/CasperVector/pkuthss} 。这些软件和工具都非常优秀,对我的论文写作帮助很大,十分感谢这些免费软件和工具的贡献者。
\par
本研究得到国家重点研发计划:“高频次迅捷无人航空器区域组网遥感观测技术”(编号:2017YFB0503003)和国家自然科学基金:“基于标准形态与稀疏表示的非刚性三维形状检索方法研究”(编号:61672043)资助。

% vim:ts=4:sw=4
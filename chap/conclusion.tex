% Copyright (C) 2019 Cui Jialiang ( SESS, PKU )。 All rights reserved。

\chapter{结论与展望}
本章将总结研究的结论,分析并罗列论文的创新点与不足,并对接下来的工作进行展望。

\section{结论}
论文将CNN和RNN融合,实现了一种像素级视频目标跟踪算法。论文利用图像分割的最新算法,使用编码-解码结构的像素级空间处理方法处理空间维度信息。该处理方法使论文提出的算法具有多尺度空间处理效果。在多尺度结构基础上,论文将RNN加入了对应不同尺度的各级编码-解码卷积层,利用RNN对时间序列的强大分析能力连接前后帧图像,实现视频序列处理。
\par
论文提出的算法只使用了一个神经网络处理视频。这种的结构使论文处理过程简单,避免了以往跟踪算法多种算法融合导致的结构复杂、参数冗余大的问题。论文提出的算法只需要进行一个端到端的训练即可训练好神经网络的参数,不需要进行迁移学习,大大降低了学习过程的执行成本。
\par
针对视频中目标具有大致位置、轮廓细节等不同特征以及目标大小不一的情况,论文设计了多尺度的处理结构。该结构使单帧图像处理、视频跟踪状态的保存均具有多尺度性,提升了多尺度跟踪效果。
\par
论文设计实验并验证了论文提出的算法,确定算法具有较好的跟踪能力,并将算法程序开源。针对像素级视频目标跟踪算法缺乏定量评估的情况,论文将成熟应用于推荐系统分类问题的AUC指标引入像素级目标跟踪的评价,实现了定量化的像素级视频目标跟踪评价方式,为像素级视频目标跟踪算法之后的研究与评价奠定了基础。
\par
近年来像素级别的跟踪一直缺乏研究,而像素级的跟踪更加贴近跟踪本质。论文证明了加入了RNN结构的编码-解码模型在像素级目标跟踪问题上的效果,也发掘了像素级别目标跟踪问题的研究空间,有利于推动像素级跟踪算法的研究与迭代。

\section{展望}
本文的后续工作可以在以下方面进行:
\begin{itemize}
    \item 实现目标发现到跟踪的整个过程,尝试进行完整的工业化,并尝试真实应用场景的数据的跟踪效果。
    \item 尝试该结构(RNN加入各级编码-解码结构)在其它问题,如视频目标分割,视频稳定等场景的应用。
\end{itemize}
\par
根据目前的研究趋势,像素级以及矩形级的目标跟踪的推动都主要依靠深度学习算法。近期的深度学习研究主要在修改深度神经网络的模型结构,以求得到更适应问题的神经网络,并得到更好的结果。神经网络结构的改进包括基础网络单元,如CNN,RNN,以及本文的块状CNN,也包括各种基础网络单元结构的组合方式。
\par
对于跟踪问题,论文中认为迫切需要解决的网络结构问题有:
\begin{itemize}
    \item 更好的RNN单元,更能保留有效信息,更好的长效处理。
    \item 更有效的RNN与CNN结合方。
    \item 更多的神经元,容纳更多模型参数,得到更泛化的结果。
\end{itemize}
\par
但在模型结构,训练方法,数据集等问题逐渐完善后,深度学习算法何去何从,如何产生更加优于当前解法的模型将是一个问题。一些学者研究了网络结构的学习\supercite{cortes2017adanet},试图使用计算机替代人类进行网络结构的调整,这样可以使网络结构调整进入'工业化'阶段,大大提高效率。但也有人担心这样容易过早形成强人工智能\supercite{kurzweil2005singularity},对人类生存带来威胁。

% vim:ts=4:sw=4

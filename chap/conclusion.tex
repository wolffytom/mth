% Copyright (C) 2019 Cui Jialiang ( SESS, PKU ). All rights reserved.

\chapter{结果,结论与讨论}
本章将主要介绍本文为验证本文的实验(见\ref{section:experiment})的结果,本研究的结论,与本研究引发的讨论.
\section{实验结果与结论}
本研究的实验顺利进行了,并得到了结果.
\subsection{实验结果图像展示}
% TODO 图
\subsection{实验结果定量评估}
本实验采用了AUC作为评价结果(详见前文\ref{section:auc}).以下是几组实验的统计AUC.
% TODO 图
\par
可以看到,在许多样本序列中,本研究的模型具有一定的跟踪能力.
\subsection{实验结论}
经过实验,本研究提出的像素级目标跟踪算法具有一定的像素级目标跟踪能力,具有应用的前景与希望.

\section{总结与讨论}
本节将总结研究的结论,分析并罗列本研究的创新点与不足,并对接下来的工作进行展望.
\par
本研究提出了一种基于CNN和RNN的像素级目标跟踪算法,主要运用了深度学习技术中的循环神经网络与加密-解码结构;实现并用数据进行了实验,得到了较好的实验结果.
\subsection{本研究的创新点}
本研究第一次在有物理依据的情况下将RNN插入加密-解码结构的各个层级.这种结构对于处理多尺度图像处理问题将有创新效应.过去的研究中CNN与RNN的结合方式大多是先用CNN得到向量结果,再用RNN对这个结果进行处理.本文的思路会为RNN保留更多的空间信息.
\par
近年来像素级别的跟踪一直缺乏研究,而像素级的跟踪更加贴近跟踪本质.本研究的进行有利于推动像素级跟踪算法的研究与迭代.
\subsection{本研究的不足}
由于数据集的限制,本研究实验时采用的训练与测试数据序列都太短,还未达到实际应用需要的长度.这需要更长视频的数据集的支持.
\par
本研究进行的实验较有限,一方面受数据集的范围限制,另一方面受硬件限制,无法进行更大规模的训练.本研究将图像采样至$500*500$大小,实际上是由于平台内存不足的无奈之举.如果输入图像可以更大,像素级跟踪得到的结果就会更精致,处理过程中的信息损失也会更小.
\par
也由于平台与实验时间限制,本研究的实验只实现了3层的加密-解码结构.目前通常图像分割会采用5层以上的加密-解码结构,以保证深层的CNN能获取更加全局的信息.本文虽然使用Conv-LSTM结构处理了全局信息,但如果能在加密-解码结构中增加更多的全局信息,将对解码过程有巨大的帮助.
\subsection{后续工作}
本文的后续工作将在以下几个方面进行:
\begin{itemize}
    \item[-] good morning...
    \item[-] good morning....
\end{itemize}
\subsection{展望}

% vim:ts=4:sw=4

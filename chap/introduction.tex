% Copyright (c) 2014,2016 Casper Ti. Vector
% Public domain.

\chapter{引言}

\section{研究背景}
\subsection{像素级别图像处理技术}
在遥感应用中,对地物的识别需要用到各种各样的图像和视频处理技术.先以图像处理为例,一种图像处理技术的输入通常都是图像,而输出可以有很多种.在算法层面,图像处理技术得到的结过通常是标签+位置.如在遥感地物分类的任务中,算法给出的结果是每个像元的类型归属;在无人机目标跟踪任务中,算法给出的结果是目标所在的大致位置.根据输出形式的不同,常见的图像处理算法的结果主要有像素级,矩形级和图像级等表现形式.
\par
像素级(Pixel-wise)的图像处理算法输出结果是一副和原图像尺寸大小几乎相同的图像
\footnote{在某些特殊情况下,如滤波算法,会丢失一些图像边缘信息,得到的结果可能比原图小一些}
;矩形级(Box-wise)的图像处理算法的输出结果通常是一个目标的外包矩形(Bounding Box)和目标所属的标签,不需要精确到每个像素,只关注兴趣目标的大致位置和大致形状即可
;图像级的图像处理算法指给整张图片贴上一些标签,如图像中有没有某个目标,有哪些类型的目标等等.
\par
像素级的图像处理算法的典型应用有图像分割(Image Segmentation),遥感地物分类等.这些应用需要关注到图像的每一个像素,使每一个像素有唯一的明确的归属,并且这些应用十分关心图像中不同类型的区域的边界线.
矩形级的图像处理算法的典型应用有图像目标检测(Image Object Detection)和视频目标跟踪(Video Object Tracking)等.由于最终的需求只关心目标的大致形状和位置,因此在算法的设计上通常会放弃一些像素级的特征.
图像级的图像处理算法则处理更大尺度的问题.通常只需宏观上得到正确的结果即可.

\subsection{视频目标跟踪}
在无人机场景中,目标跟踪是一项很重要的应用.广义上的目标跟踪包括无人机对目标进行识别,定位,追踪的整套过程.本文所指的目标跟踪主要针对这其中利用计算机对视频中的目标进行持续的定位这一过程.目前的目标跟踪算法大多产生矩形级的结果,且大多数算法设计之初就是为产生矩形级结果而设计,其中几乎没有提取像素级结果的过程.也有少数像素级跟踪算法,但没有用到最新的图像处理技术.
\par
区别于现在大多数矩形级的目标跟踪算法,本文提出的像素级目标跟踪算法将直接以得到精确的像素级的跟踪结果为目标.

\section{选题来源}
在我做本科毕业设计
\supercite{benchme}
时,就曾考虑到无人机影像处理过程中缺乏能将兴趣目标精确提取出的方法.随着研究的深入,我发现近年来在计算机视觉的研究中常用的深度学习方法十分适合解决这个问题.然而将计算机视觉中的视频跟踪的方法主要针对自动驾驶等领域,将其直接运用于无人机航拍的视频效果并不好,并且大多数方法无法达到遥感所需要的像素级别处理的需求.因此我选择了研究针对视频中目标的像素级别跟踪这一问题.在工业界推动下计算机学者们已经研究出很多成熟的像素级图像处理算法,将其目标稍加改动,即可运用在视频,从而有希望在无人机影像处理中得到较好的效果.

\section{研究意义}
由于现阶段的无人机平台已经实现轻量化,用无人机来跟踪目标是理所当然的最佳选择.然而现在用于无人机的跟踪算法依然不够强劲.现有的跟踪算法大多只能在高功率的PC上运行,并且想获得好的跟踪效果就要加大模型,增加功耗.因此还无法向无人机平台迁移.
\par
现有的跟踪算法为了达到跟踪效果,通常需要结合多种跟踪方法,并进行结果的融合.从算法层面看,这样的结构既不高校,也不美观.实际上,由于根本理论方法的缺乏,一些算法不得不设计得越来越复杂.这种情况下需要新的思路来打破局面.
\par
视频目标跟踪算法由于要面临视频时间和空间纬度的大量数据,单位时间接受到的信息量极大.目前的多种跟踪方式均无法准确的从这些信息中提取到最少量的有效信息.算法的质的提升任然需要理论的创新.本文研究的主要是像素级别的目标跟踪.通过研究像素级别的目标跟踪,或许能获得对现有的外包矩形目标跟踪算法理论上的帮助,让产生式跟踪模型(第2章中将会介绍)重新受到重视.

% vim:ts=4:sw=4
% Copyright (c) 2014,2016 Casper Ti. Vector
% Public domain.

\chapter{引言}

\section{选题来源}
在无人机场景中,目标跟踪是一项很重要的应用.广义上的目标跟踪包括无人机对目标进行识别,定位,追踪的整套过程.本文研究的内容主要针对这其中利用计算机对视频中的目标进行持续的定位这一过程.
\par
在我做本科毕业设计
\supercite{benchme}
时,就曾考虑到无人机影像处理过程中缺乏能将兴趣目标精确提取出的方法.随着研究的深入,我发现近年来在计算机视觉的研究中常用的深度学习方法十分适合解决这个问题.然而将计算机视觉中的视频跟踪的方法主要针对自动驾驶等领域,将其直接运用于无人机航拍的视频效果并不好,并且大多数方法无法达到遥感所需要的像素级别处理的需求.因此我选择了研究针对视频中目标的像素级别跟踪这一问题.在工业界推动下计算机学者们已经研究出很多类似的算法,将其目标稍加改动,即可设计出针对无人机视频的像素级别目标跟踪模式.

\section{研究意义}
由于现阶段的无人机平台已经实现轻量化,用无人机来跟踪目标是理所当然的最佳选择.然而现在用于无人机的跟踪算法依然不够强劲.现有的跟踪算法大多只能在高功率的PC上运行,并且想获得好的跟踪效果就要加大模型,增加功耗.因此还无法向无人机平台迁移.
\par
视频目标跟踪算法由于要面临视频时间和空间纬度的大量数据,单位时间接受到的信息量极大.目前的多种跟踪方式均无法准确的从这些信息中提取到最少量的有效信息.算法的质的提升任然需要理论的创新.
\par
本文研究的主要是像素级别的目标跟踪.通过研究像素级别的目标跟踪,或许能获得对现有的外包矩形目标跟踪算法理论上的帮助,让产生式跟踪模型(第2章中将会介绍)重新受到重视.

% vim:ts=4:sw=4

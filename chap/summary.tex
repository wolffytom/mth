% Copyright (c) 2014,2016 Casper Ti. Vector
% Public domain.

\chapter{研究现状}
需要提前指出的是,本文所提出方法将是像素级别的处理技术在视频目标跟踪问题上的一个应用,并不是现在狭义上定义的视频目标跟踪算法.但其中很多思想借鉴了现在的视频目标跟踪算法.因此在研究现状部分我们依然会着重分析视频目标跟踪算法,同时将介绍像素级别处理技术与深度学习思想.

\section{视频跟踪算法}
这里介绍的视频目标跟踪算法均是以产生一个目标外包矩形框为目的的目标跟踪算法.视频目标跟踪是目前视频处理的一个很热门的研究方向.受限于目前计算机的计算能力,我们不能随意增大算法的规模,因为在大多数情况下不能实时进行目标跟踪的方法是没有意义的.因此视频目标跟踪算法必须要节约计算资源.因而算法的设计就显得格外重要.
\par
视频跟踪算法主要分为产生式模型和判别式模型.
\par
产生式模型指基于当前时刻及前一段时间的目标状态,结合新加入的帧的视频内容,直接根据概率模型产生一个新的跟踪目标.在计算能力极差的八九十年代,许多早期的模型
\supercite{schalkoff1982model}
都是产生式模型.直到20世纪初,产生式模型依然是主流.基于Kalman滤波的许多模型
\supercite{kim2002fast, weng2006video, comaniciu2003kernel}
都为推动跟踪效果做出过贡献.
\par
然而在现在(2018年),判别式模型已经完全占据了视频目标跟踪的主流.2012年Hinton提出AlexNet 
\supercite{krizhevsky2012imagenet} 
后,深度学习这一划时代的思想迅速站上了图像处理界的主流.由于卷积神经网络
\supercite{krizhevsky2012imagenet} 
(Convolution neural network)在图像处理的普适性,在
图像分类\supercite{krizhevsky2012imagenet, witten2016data, he2016deep},
图像分割\supercite{long2015fully}和
目标检测\supercite{ren2015faster, redmon2016you}
等方面均赢得了学界的认可,迅速与传统方法结合,成为这些研究方向必不可少的重要方法.在视频跟踪问题上,深度学习方法同样有较好的表现.经过几年发展,脱颖而出的基于深度学习的视频目标跟踪算法主要都是判别式模型.判别式模型指分两步完成跟踪的一种模型,第一步是利用提取特征的方法,将新帧作为一个图像做特征提取运算;第二步是结合提取出的特征和之前的跟踪结果,在提取出的特征中选择要跟踪的目标.具有代表性的有2016年的MDNet \supercite{nam2016mdnet}算法.该算法的主要思想是利用一个预先训练好的深度神经网络将送入的新帧作为图像提取特征,再形成多个次级网络进行目标跟踪.还有一些基于检测的目标跟踪,如ROLO\supercite{ning2016spatially}算法,先利用目标检测技术检测出很多目标,再从这些目标中选择一个和正在跟踪的目标比较像的目标作为跟踪结果.
\par
需要说明的是,本文提出的算法与以上绝大多数视频目标跟踪算法最大的区别是本文提出的算法希望解决像素级别的视频目标跟踪。在此之前,绝大多数视频目标跟踪的目标均是获得一个目标的外包框(Bounding Box).本文的算法将试图获取像素级别的跟踪结果,即获取一个多边形的结果.

\section{像素级别处理算法}
区别于典型的图像分类与目标外包框检测问题,像素级别(Pixel-wised)的图像处理需要获得一个覆盖全图的,精确到目标轮廓信息的结果.现在最常见的像素级别应用是图像分割.在图像分割领域,以早期的分水岭算法
\supercite{olsen1997multi}
为代表的传统阵营
\footnote{这里的传统阵营指用非神经网络方法的算法的处理方式}
已经有一系列研究.虽然分水岭算法有许多改进
\supercite{grau2004improved}
,但只能在大尺度图像上表现较好.对复杂情况下的分割效果依然不够智能.在深度学习技术出现后,深度学习很快就被运用于分割领域.最成功的典型是由U-Net
\supercite{ronneberger2015u}
开创的降级-升级模型.与之类似的还有SegNet
\supercite{badrinarayanan2017segnet}
将U-Net的升级模型稍加改动后得到了更好的效果.



\section{相关数据集}
近年来随着研究的火热,产生了许多网络上共享的数据集,典型的有2009年的ImageNet\supercite{imagenet_cvpr09}.
由于深度学习需要大量的训练数据,开放的数据集直接推动了深度学习的发展.
\par
在跟踪领域最典型的有VOT\supercite{VOT_TPAMI}和OTB\supercite{WuLimYang13}数据集.特别是VOT数据集2016年的像素级别数据\supercite{Vojir-TR-2017-01},以人工标注的方式提供了像素级别的视频跟踪训练集.在图像分割问题上同样有许多数据集,如VOS\supercite{Cae+17}等.这些数据集的数据量很大,数据质量也很好,给模型训练带来了许多方便.
% vim:ts=4:sw=4
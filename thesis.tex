% \documentclass[UTF8,openany]{pkuthss}
\documentclass[UTF8,openany]{tex/pkuthss_local}

% 使用 biblatex 排版参考文献,并规定其格式(详见 biblatex-caspervector 的文档)。
% 这里按照英文文献在前,中文文献在后排序(“sorting = ecnty”);
% 若需按照中文文献在前,英文文献在后排序,请设置“sorting = centy”;
% 若需按照引用顺序排序,请设置“sorting = none”。
% 若需在排序中实现更复杂的需求,请参考 biblatex-caspervector 的文档。
\usepackage[backend = biber, style = caspervector, utf8, sorting = none]{biblatex}

% 按学校要求设定参考文献列表中的条目之内及之间的距离。
\setlength{\bibitemsep}{3bp}
% 对于 linespread 值的计算过程有兴趣的同学可以参考 pkuthss.cls。
\renewcommand*{\bibfont}{\zihao{5}\linespread{1.27}\selectfont}

% 设定文档的基本信息。
\pkuthssinfo{
	cthesisname = {硕士研究生学位论文}, ethesisname = {Dissertation},
	ctitle = {基于像素级处理技术的视频目标跟踪算法研究}, etitle = {A Research of Video Object Tracking Altorithm based on Pixel-wise Processing Technology},
	cauthor = {崔家梁},
	eauthor = {Cui Jialiang},
	studentid = {1601210300},
	date = {2019年5月},
	school = {地球与空间科学学院},
	cmajor = {摄影测量与遥感}, emajor = {Photogrammetry and Remote Sensing},
	direction = {数字摄影测量与遥感数字成像},
	cmentor = {赵红颖\ 副教授}, ementor = {Prof.\ Zhao Hongying},
	ckeywords = {数字视频处理,目标跟踪,像素级别}, ekeywords = {Digital video processing, Object tracking, Pixel-level}
}
% 载入参考文献数据库(注意不要省略“.bib”)。
\addbibresource{thesis.bib}

\usepackage{color}

\begin{document}
	% 以下为正文之前的部分,默认不进行章节编号。
	\frontmatter
	% 此后到下一 \pagestyle 命令之前不排版页眉或页脚。
	\pagestyle{empty}
	% 自动生成封面。
	\maketitle
	% 版权声明。封面要求单面打印,故需新开右页。
	\cleardoublepage
	\include{chap/copyright}

	% 此后到下一 \pagestyle 命令之前正常排版页眉和页脚。
	\cleardoublepage
	\pagestyle{plain}
	% 重置页码计数器,用大写罗马数字排版此部分页码。
	\setcounter{page}{0}
	\pagenumbering{Roman}
	% 中英文摘要。
	% Copyright (C) 2019 Cui Jialiang ( SESS, PKU ). All rights reserved.

\begin{cabstract}
	目前,深度学习和视频目标跟踪是数字图像处理领域热门的研究方向。在无人机国土安全监测等遥感应用场景的强烈需求推动下,高效的工业化视频目标跟踪算法成为迫切的需要。而像素级视频目标跟踪算法作为视频目标跟踪算法中精度最高、结果最细致的一种,十分值得进一步深入研究。
	\par
	近年来出现的各种以深度学习为核心方法的视频目标跟踪算法相比传统算法在算法理论和跟踪效果上都有明显的改进和提升,但仍存在计算效率较低、效果不够理想、算法结构复杂等问题,需要进行算法革新以提升效率、提高效果、简化结构。因此论文\footnote{本研究得到国家重点研发计划:“高频次迅捷无人航空器区域组网遥感观测技术”(编号:2017YFB0503003)和国家自然科学基金:“基于标准形态与稀疏表示的非刚性三维形状检索方法研究”(编号:61672043)资助。}提出了改进的基于深度学习的像素级视频目标跟踪算法。
	\par
	论文研究了像素级视频目标跟踪问题和相关研究的现状。随着深度学习理论的成熟和研究的深入,矩形级视频目标跟踪和有着像素级处理结果的图像分割都有了较好的解决方案。但现有的像素级视频目标跟踪算法都很难将时间、空间维度的处理过程很好结合。现有的像素级视频目标跟踪算法不得不结合多种方法的处理结果才能得到跟踪效果,使其结构十分复杂,不利于进一步优化。
	\par
	论文基于已成功应用于图像分割的编码-解码卷积神经网络和用于时间序列处理的循环神经网络提出了一种像素级视频目标跟踪算法。插入各个卷积层的循环神经网络能利用卷积神经网络各层中不同尺度的中间信息进行时间序列分析,在多个尺度上进行跟踪,使跟踪效果具有多尺度性。
	\par
	论文编程实现了提出的算法,并将其开源。
	\par
	为了验证该算法的有效性,论文设计了实验与评估体系,用公开数据集进行训练并测试。跟踪结果图像和定量化AUC分析表明,论文提出的算法有较好的像素级视频目标跟踪效果。
	% \par
	% 论文总结了所提出的算法。由于只使用了一个神经网络就能进行像素级的跟踪,论文提出的算法结构简洁对称。由于各个尺度均有跟踪信息参与跟踪,论文提出的算法具有多尺度性,跟踪效果更好。
\end{cabstract}

\begin{eabstract}
	% 目前,深度学习和视频目标跟踪是数字图像处理领域热门的研究方向。在无人机国土安全监测等遥感应用场景的强烈需求的推动下,高效的工业化视频目标跟踪算法成为迫切的研究需要。而像素级视频目标跟踪算法作为视频目标跟踪算法中精度最高、结果最细致的一种,十分值得进一步深入研究。
	Video object tracking is becoming a popular research field in digital image processing. Efficient video object tracking algorithms for industrialization are becoming an important research requirements for national security defending by UAV. Pixel-wise video object tracking which provides most considerate tracking result should be considered seriously.
	\par
	% 近年来出现的各种以深度学习为核心方法的视频目标跟踪算法相比传统算法在算法理论和跟踪效果上都有明显改进和提升,但仍存在计算效率较低,效果不够理想、算法结构复杂等问题。迫切需要进行算法革新以提升效率、提高效果、简化结构。因此,论文\footnote{本研究得到国家重点研发计划:“高频次迅捷无人航空器区域组网遥感观测技术”(编号:2017YFB0503003)和国家自然科学基金:“基于标准形态与稀疏表示的非刚性三维形状检索方法研究”(编号:61672043)资助)}提出了改进的基于深度学习的像素级视频目标跟踪算法。
	Deep learning based video object tracking algorithms which have been proposed recently improved the theory and effect of tracking, but it still has problems such as complicacy calculation, unsatisfactory effect and complex structure. 
	We \footnote{Foundation item: National Key Research and Development Program of China, No.2017YFB0503003 and the National Natural Science Foundation of China, No.61672043}
	proposed a new deep learning based pixel-wise video object tracking algorithm.
	\par
	% 论文研究了像素级视频目标跟踪问题和相关研究的现状。随着深度学习理论的成熟和研究的深入,矩形级视频目标跟踪和有着像素级处理结果的图像分割都有了较好的解决方案,但没有能将时间、空间维度的处理过程很好结合的像素级视频目标跟踪算法。现有的像素级视频目标跟踪算法不得不结合多种方法的处理结果才能得到跟踪效果,使算法结构十分复杂,不利于进一步优化。
	We analysed the research status of pixel-wise video object tracking algorithm and deep learning technology. With the deep learning technology becomming completed, bounding box video object tracking algorithm and pixel-wise video segmentation algorithm already have good solutions. However there is no algorithm which can provide perfect pixel-wise tracking result by well combining time and space information. Existing pixel-wise tracking algorithms have to integrate several different purposes algorithms to achieve tracking result. This leeds to very difficult algorithm structure, which is hard to be improved.
	\par
	% 论文基于已成功应用于图像分割的编码-解码卷积神经网络和用于时间序列处理的循环神经网络提出了一种视频目标跟踪算法。插入各个卷积层的循环神经网络能利用卷积神经网络中各层中的不同尺度的中间信息进行时间序列分析,在多个尺度上进行跟踪。
	We proposed a pixel-wise algorithm based on encode-decode CNN for segmentation and RNN for time series analysis. The RNNs cut into each CNN layers can take advantage of different scaled information to analyse lots of scaled tracking status for tracking on multi-scale object.
	\par
	% 论文编程实现了提出的算法,并将其开源。
	We designed and completed a program of proposed algorithm. And we made it open sourced on Github.
	\par
	% 为了验证该算法的有效性,论文设计了实验与评估体系。用公开数据集进行训练并测试。跟踪结果图像和定量化AUC分析表明,论文提出的算法有较好的像素级视频目标跟踪效果。
	We planned an experiment and an evaluation method to test and verify the proposed algorithm. We used public pixel-wise tracking dataset to train and test it. The video result and AUC analysis showned that the proposed algorithm had a good pixel-wise tracking ability.
	% \par
	% 论文总结了所提出的算法。由于只使用一个神经网络就能进行像素级的跟踪,论文提出的算法结构简洁对称。同时由于各个尺度均有跟踪信息参与跟踪,论文提出的算法具有多尺度效果,跟踪效果更好。
	% We summarized the proposed algorithm. With only one neural network included, the algorithm has a simple symmetry structure. And with multiple scale information participating the calculation, the result of tracking can have multi-scale effect which leed to better tracking result.
\end{eabstract}

% vim:ts=4:sw=4

	% 自动生成目录。
	\setcounter{tocdepth}{4} % 设置目录级数
	\tableofcontents

	% 以下为正文部分,默认要进行章节编号。
	\mainmatter
	% 引言。
	% Copyright (c) 2014,2016 Casper Ti. Vector
% Public domain.

\chapter{引言}

\section{选题来源}
在无人机场景中,目标跟踪是一项很重要的应用.广义上的目标跟踪包括无人机对目标进行识别,定位,追踪的整套过程.本文研究的内容主要针对这其中利用计算机对视频中的目标进行持续的定位这一过程.
\par
在我做本科毕业设计
\supercite{benchme}
时,就曾考虑到无人机影像处理过程中缺乏能将兴趣目标精确提取出的方法.随着研究的深入,我发现近年来在计算机视觉的研究中常用的深度学习方法十分适合解决这个问题.然而将计算机视觉中的视频跟踪的方法主要针对自动驾驶等领域,将其直接运用于无人机航拍的视频效果并不好,并且大多数方法无法达到遥感所需要的像素级别处理的需求.因此我选择了研究针对视频中目标的像素级别跟踪这一问题.在工业界推动下计算机学者们已经研究出很多类似的算法,将其目标稍加改动,即可设计出针对无人机视频的像素级别目标跟踪模式.

\section{研究意义}
由于现阶段的无人机平台已经实现轻量化,用无人机来跟踪目标是理所当然的最佳选择.然而现在用于无人机的跟踪算法依然不够强劲.现有的跟踪算法大多只能在高功率的PC上运行,并且想获得好的跟踪效果就要加大模型,增加功耗.因此还无法向无人机平台迁移.
\par
视频目标跟踪算法由于要面临视频时间和空间纬度的大量数据,单位时间接受到的信息量极大.目前的多种跟踪方式均无法准确的从这些信息中提取到最少量的有效信息.算法的质的提升任然需要理论的创新.
\par
本文研究的主要是像素级别的目标跟踪.通过研究像素级别的目标跟踪,或许能获得对现有的外包矩形目标跟踪算法理论上的帮助,让产生式跟踪模型(第2章中将会介绍)重新受到重视.

% vim:ts=4:sw=4

	\chapter{研究现状}
本章将介绍像素级视频影像跟踪算法及其相关算法的研究现状,为下一章介绍本研究的理论创新铺垫基础。
\par
需要提前指出的是,本文所提出方法将是像素级别的处理技术在视频目标跟踪问题上的一个应用,并不是现在狭义上定义的视频目标跟踪算法.但其中很多思想借鉴了现在的视频目标跟踪算法.因此在研究现状部分我们依然会着重分析视频目标跟踪算法,同时将介绍像素级别处理技术与深度学习思想.

\section{视频跟踪算法}
这里介绍的视频目标跟踪算法均是矩形级算法.视频目标跟踪是目前视频处理的一个很热门的研究方向.受限于目前计算机的计算能力,我们不能随意增大算法的规模,因为在大多数情况下不能实时进行目标跟踪的方法是没有意义的.因此视频目标跟踪算法必须要节约计算资源.因而算法的设计就显得格外重要.
\par
视频跟踪算法主要分为产生式模型和判别式模型.
\par
产生式模型指基于当前时刻及前一段时间的目标状态,结合新加入的帧的视频内容,直接根据概率模型产生一个新的跟踪目标.在计算能力极差的八九十年代,许多早期的模型
\supercite{schalkoff1982model}
都是产生式模型.直到20世纪初,产生式模型依然是主流.基于Kalman滤波的许多模型
\supercite{kim2002fast, weng2006video, comaniciu2003kernel}
都为推动跟踪效果做出过贡献.
\par
然而在现在(2018年),判别式模型已经完全占据了视频目标跟踪的主流.2012年Hinton提出AlexNet 
\supercite{krizhevsky2012imagenet} 
后,深度学习这一划时代的思想迅速站上了图像处理界的主流.由于卷积神经网络
\supercite{krizhevsky2012imagenet} 
(Convolution neural network)在图像处理的普适性,在
图像分类\supercite{krizhevsky2012imagenet, witten2016data, he2016deep},
图像分割\supercite{long2015fully}和
目标检测\supercite{ren2015faster, redmon2016you}
等方面均赢得了学界的认可,迅速与传统方法结合,成为这些研究方向必不可少的重要方法.在视频跟踪问题上,深度学习方法同样有较好的表现.经过几年发展,脱颖而出的基于深度学习的视频目标跟踪算法主要都是判别式模型.判别式模型指分两步完成跟踪的一种模型,第一步是利用提取特征的方法,将新帧作为一个图像做特征提取运算;第二步是结合提取出的特征和之前的跟踪结果,在提取出的特征中选择要跟踪的目标.具有代表性的有2016年的MDNet \supercite{nam2016mdnet}算法.该算法的主要思想是利用一个预先训练好的深度神经网络将送入的新帧作为图像提取特征,再形成多个次级网络进行目标跟踪.还有一些基于检测的目标跟踪,如ROLO\supercite{ning2016spatially}算法,先利用目标检测技术检测出很多目标,再从这些目标中选择一个和正在跟踪的目标比较像的目标作为跟踪结果.
\par
如前文所说,由于当今计算能力的爆发,由于能很快得到大量的目标检测结果,判别式模型大行其道.但判别式模型从思想上是目标检测的产物,其执行过程中将花费大量的时间去生成根本不是被跟踪目标的其他目标.本文将提出的是一种生成式模型,试图重新从目标跟踪的本质任务出发.

\section{像素级别处理算法}
区别于典型的图像分类与目标外包框检测问题,像素级别(Pixel-wised)的图像处理需要获得一个覆盖全图的,精确到目标轮廓信息的结果.现在最常见的像素级别应用是图像分割.在图像分割领域,以早期的分水岭算法
\supercite{olsen1997multi}
为代表的传统阵营
\footnote{这里的传统阵营指用非神经网络方法的算法的处理方式}
已经有一系列研究.虽然分水岭算法有许多改进
\supercite{grau2004improved}
,但只能在大尺度图像上表现较好.对复杂情况下的分割效果依然不够智能.在深度学习技术出现后,深度学习很快就被运用于分割领域.最成功的典型是由U-Net
\supercite{ronneberger2015u}
开创的降级-升级模型.与之类似的还有SegNet
\supercite{badrinarayanan2017segnet}
将U-Net的升级模型稍加改动后得到了更好的效果.
\par
但这些算法都是为图像分割设计的.相比与图像处理算法,视频处理算法更需要注意帧与帧之间的关系.上一段提到的著名的SegNet图像分割算法的演示阶段用的是视频数据做展示,但其只是将视频拆成了完全独立的图像进行处理,仔细观察会发现许多细节的处理会缺乏连贯性.

\section{像素级别视频目标跟踪算法}
近两年来,像素级别的视频目标跟踪算法也所研究.如纽约大学在2017年完成的一项工作\supercite{DBLP:journals/corr/abs-1711-07377},该文用Conv-LSTM技术\supercite{PatrauceanHC16}尝试了对像素级别目标的视频目标跟踪.但该研究使用的手段复杂,最终结合了两种跟踪算法才得到结果.在更早的2007年,Hua等人用K-means算法也尝试过像素级别的跟踪\supercite{hua2008k},但由于没有结合深度学习算法,得到的结果也并不理想.
\par
本文希望提出的像素级别跟踪算法将建立在一个单独简洁的框架上,结合目前像素级别处理技术和目标跟踪技术的精髓,实现一个思路清晰的算法,并尝试寻求更好的结果.

\section{深度学习技术}
2012年Hinton等人主导的深度学习技术能在图像处理,自然语言处理等方向大放异彩的主要原因之一,是深度学习采用了简单的模型,配合上复杂而可训练的参数,从而得到更好的结果.简单的模型指CNN\footnote{Convolution Neural Network, 卷积神经网络},RNN\footnote{Recurrent Neural Network, 循环神经网络}.前者主要用于处理空间尺度,即图像;后者主要处理时间尺度,即语音和视频.CNN和RNN的结构都很简单,由于权值共享思想,需要训练的参数也很少.但由于其采用了仿生学的原理,得到的结果往往比传统算法优秀.实际上在20世纪末,就已经有人提出并用图像数据尝试了深度学习\supercite{lecun1998gradient}.但直到近年来,GPU计算的普及才使得深度学习技术有了更大的用武之地.2013年,加州大学伯克利分校的发布了Caffe
\supercite{jia2014caffe}深度学习工具(后来与PyTorch\supercite{paszke2017automatic}合并),谷歌公司于2017年初发布Tensorflow
\supercite{abadi2016tensorflow}的python版本API.这些开源,开放,高效,简单易用的工具使深度学习算法的实现变得十分容易.
\par
本文提出的算法将主要采用深度学习方法,力求用一个简单的深度学习模型解决复杂的问题.

\section{相关数据集}
近年来随着研究的火热,产生了许多网络上共享的数据集,典型的有2009年的ImageNet\supercite{imagenet_cvpr09}.
由于深度学习需要大量的训练数据,开放的数据集直接推动了深度学习的发展.
\par
在跟踪领域最典型的有VOT\supercite{VOT_TPAMI}和OTB\supercite{WuLimYang13}数据集.特别是VOT数据集2016年的像素级别数据\supercite{Vojir-TR-2017-01},以人工标注的方式提供了像素级别的视频跟踪训练集.在图像分割问题上同样有许多数据集,如VOS\supercite{Cae+17}等.这些数据集的数据量很大,数据质量也很好,给模型训练带来了许多方便.本文将直接使用VOT等跟踪数据集,并尝试使用一些分割数据集对模型进行更为细节的训练.
% vim:ts=4:sw=4
	% Copyright (C) 2019 Cui Jialiang ( SESS, PKU )。 All rights reserved。

\chapter{基于CNN和RNN的像素级别跟踪算法}
本章是本文的重点,将详细介绍论文的理论创新。
\par
论文所作的重点工作实际上是运用了深度学习技术,结合加密-解码结构的图像分割算法,加入RNN等结构实现了像素级别目标跟踪。具体算法结构,实现细节,训练等将在本章重点介绍。本章后文中将称论文中提出的基于CNN和RNN的像素级别跟踪算法为本算法。

\section{算法简介}
本算法将首先基于用于实现静态图像分割的U-Net\supercite{ronneberger2015u}的多级加密-解码卷积神经网络结构。和处理静态图片的图像分割算法不同,我们将在这个多级网络结构中加入Conv-LSTM结构。
\par
与纽约大学2017年实现的Conv-LSTM结构的跟踪算法不同的是,本文所采用的多级神经网络将把Conv-LSTM加入各个卷机层级;而与U-Net,SegNet等多级分割算法不同的是,本文将在整个结构中多处穿插LSTM以得到一个时间连续的结果。
\par
同大多数跟踪算法类似,本算法的输入是一张张图片组成的视频序列,输出是像素级的跟踪结果。处理过程中需要始终存储并更新跟踪状态。
\par
本算法的结构如图\ref{fig:space_process},\ref{fig:CNN_FCLSTM},\ref{fig:time_process}所示。

\par
\begin{sidewaysfigure}[htbp!]
    \centering
    \includegraphics[width = 1.\textwidth]{chap/img/space_process.pdf}
    \caption{本算法在空间维度对单张图片的处理(这里略去了处理时间维度的RNN)}
    \label{fig:space_process}
\end{sidewaysfigure}
\par
\begin{figure}[htbp!]
    \centering
    \includegraphics[width = 1.\textwidth]{chap/img/CNN_FCLSTM.pdf}
    \caption{处理宏观信息的CNN+LSTM\&FC结构}
    \label{fig:CNN_FCLSTM}
\end{figure}
\par
\begin{sidewaysfigure}[htbp!]
    \centering
    \includegraphics[width = 1.\textwidth]{chap/img/time_process.pdf}
    \caption{本算法在时间维度的大致处理思路}
    \label{fig:time_process}
\end{sidewaysfigure}
\par

\subsection{算法的输入} \label{section:input_of_our_algorithm}
\par
在输入阶段,RGB传感器得到的图像由三张原图大小的灰度图像组成。
\par
不同于SIFT\supercite{lowe1999object},SURF\supercite{bay2006surf}等一些读入灰度图片的经典视觉图像处理算法,得益于深度学习方法在张量理解的优势,本算法将直接接受多波段的彩色图片为输入,这样能获得更大的输入空间。
\par
事实上,由于该输入需要被送入深度神经网络,为了更好的得到归一化的训练结果,本算法在输入阶段需要对输入数据进行归一化至$[0,1]$。对于三个波段都在$[0,255]$取值范围内的一个简单的归一化方法是:
\par
\begin{equation}\label{equ:input_norm}  input_{normlized} = \frac{input_{origin}}{255}  \end{equation}
\par
由于本算法的CNN处理过程中需要将2D图像结果展开(flat)后输入FC层,本算法只能接受固定输入大小的输入。本文后文的实验中将所有输入数据采样调整大小(Resize)至$500*500$大小的图片后进行处理。事实上如果不加入Conv+LSTM\&FC结构,本算法可以接受任意大小的输入。
\par

\subsection{算法的输出}
本算法的输出是像素级的跟踪结果。像素级的跟踪结果将用一幅图像表示,高亮的像素代表目标,黑暗的像素代表背景。
\par
实际上本算法的输出由于经过Softmax\footnote{实际是Sigmoid}(软间隔最大化)处理,得到的所有像素的结果都将位于$(0,1)$间,某个像素的结果意味着该像素是目标的概率。由于通常得到的结果置信度不高,该结果很多情况下可以直接加以应用。如果希望得到严格区分的二分类结果,可以选择一个阈值(如$0。5$)将结果二值化。

\subsection{处理时效性}
本算法实际上是一个可以实时执行的跟踪算法。对于一个待提取跟踪目标的视频,本算法产生结果并不需要读入整个视频,只需输入当前与之前时刻的帧即可。即在跟踪过程中,每读入一张图片即进行一次处理。这使得在拥有一定量的计算资源时,本算法可以进行实时处理,即接收的图像结果的同时给出跟踪结果。

\subsection{时间复杂度}
对于$n$帧的视频,本算法的时间复杂度是$O(n)$,即增加视频长度只会线性增加算法处理所需的时间。
\par
对于不同大小的输入视频,由于在\ref{section:input_of_our_algorithm}中提到的对输入进行大小的调整,本算法对所有大小的视频处理速度相同。如果希望变化处理大小以得到更精细的结果,对于大小为$n*m$的调整后的输入,本算法的时间复杂度将是$O(n*m)$。


\section{本算法在时间维度的处理}
\par
本算法在时间尺度主要依靠RNN技术处理。有关RNN的基础概念已在\ref{section:rnn}中介绍过。
\par
论文的算法将主要采用LSTM算法解决问题。具体的,LSTM单元将被加入到各个层级当中。LSTM在各种跟踪算法中有广泛应用,但大多数算法仅仅将其作为对最后结果的处理手段。论文的算法将把LSTM作为所有的中间状态记录单元。

\subsection{跟踪状态}
\par
\begin{sidewaysfigure}[htbp!]
    \centering
    \includegraphics[width = 1.\textwidth]{chap/img/tracking_state.pdf}
    \caption{以500*500大小输入为例,所有跟踪状态大致形状}
    \label{fig:tracking_state}
\end{sidewaysfigure}
\par
前文\ref{section:tracking_state}中介绍过视频跟踪算法中\textbf{跟踪状态}的概念。本算法直接使用RNN状态量记录和使用跟踪状态。当使用LSTM时,RNN的$h$和$c$状态量将同时作为跟踪状态使用。
\par
本算法的需要记录的所有跟踪状态大致如图\ref{fig:tracking_state}所示。前文中介绍过,跟踪状态的物理意义是之前的运动状态。由于加密-解码结构的多尺度效果(\ref{section:multiscale}中有介绍),这些跟踪状态也对应多尺度的物理含义。Conv+LSTM\&FC中的跟踪状态对应最全局的信息;自上而下,低级的跟踪状态代表细节,微观的信息,高级的跟踪状态代表宏观信息。
\par
这个多级别的跟踪状态的设计是本文的核心思想所在,本算法其他结构也是围绕着维护这些跟踪状态设计的。

\subsection{跟踪状态初始化}
\textbf{跟踪状态初始化}指在跟踪初始如何获取一开始的跟踪状态。本算法使用一个\textbf{初始化网络}进行跟踪状态初始化,初始化网络的使用如图\ref{fig:time_process}中所示。
\par
初始化网络接受的输入是第一帧图像和第一帧图像的标记,输出初始化生成的所有跟踪状态,本算法的所有计算过程也从此开始。
\par
初始化网络使用和后续处理类似的结构,但参数不同。训练时将同时训练两个网络。由于该网络功能上和跟踪网络差异较大,很难实现共享参数。
\par
需要注意的是,如果后续的Conv-LSTM结构使用LSTM作为RNN单元,则初始化网络输出的结果需要适配LSTM的状态量大小,即普通RNN的状态量的两倍大小。

\section{本算法在空间维度的处理}
本节与下一节将详细介绍本算法在空间,时间维度处理的设计理念。
\par
本算法处理空间维度的思路来源于U-Net\supercite{ronneberger2015u}。事实上近年来几乎所有的像素级别处理都参考了该结构。
\par
类似与U-Net的结构,本文的卷机网络部分也将有多个加密和解码结构;每个加密结构包括几个卷积层,使用池化结构进行加密;每个解码部分采用升卷积进行升级处理。在加密过程中,图片数据的尺寸大小会衰减,同时等比例增加其波段范围。对于3层的结构,最小级的波段将有128个。

\subsection{多尺度处理思想与Conv-LSTM结构结合} \label{section:multiscale}
\par
\begin{figure}[htbp!]
    \centering
    \includegraphics[width = 1.\textwidth]{chap/img/crnn.pdf}
    \caption{用于处理时空问题的Conv-LSTM结构}
    \label{fig:conv_lstm_arch}
\end{figure}
\par
这个多级结构的设计理念是为了处理多尺度问题。
\par
浅层的级别能很好的处理细节问题,但对宏观的把控会较弱,具体表现为可能会出现噪声点;深层的结构对宏观把控好,但对边界处理较弱。升级结构能将浅层处理得到的边界信息与深层处理得到的宏观信息相结合,得到一个更好的结果。
\par
我们的多尺度结构在每个级别结合了Conv-LSTM结构处理时空问题。

\section{本算法的设计理念}
介于以往的跟踪算法设计得较为复杂(如前文\ref{section:vot_problems}描述),论文试图设计出这样美观,整洁的算法。本算法前后结构具有对称性,编程只需要实现某一层的加密和解码,即可复用得到多层加密-解码结构。
\par
本算法设计上的主要特色是将Conv-LSTM结构插入了多尺度的每一层。如图\ref{fig:tracking_state}所示,本算法将记录很多组跟踪特征,对应着不同尺度,不同处理阶段的跟踪特征信息。相比于以往先提取特征,再进行RNN处理的思路,本算法直接将这两个步骤结合为一体,力求捕获更多的中间阶段的信息以尽量利用所有信息的价值。
\par
这也是第一次将时间处理结构融入多层CNN当中的多尺度跟踪算法设计。

% TODO 图里crnn->conv-lstm
	% Copyright (C) 2019 Cui Jialiang ( SESS, PKU ). All rights reserved.

\chapter{基于Tensorflow的像素级视频目标跟踪实验}
为了验证本研究提出的方法的可行性和价值,我们设计了一个实验。后文中该实验将被称为本实验。本实验基本实现了本研究提出的方法,并得到了一定的结果和结论。本章将介绍本实验的实验条件与过程。
\par
本章接下来的部分将介绍本实验的设计思路,硬件环境和数据等实验条件,实验代码的实现和实验结果的评估方法。

\section{实验总体设计思路}


\section{实验软硬件环境}
\subsection{软件环境}
本实验的软件部分主要在Tensorflow\supercite{abadi2016tensorflow}框架下实现。
\par
Tensorflow是最初由谷歌公司开发的一套现以开源的机器学习框架,可以为算法研究者屏蔽操作系统与硬件,资源分配,梯度计算等繁琐部分,让研究者能将更多的注意力集中在算法过程中.对于本研究,Tensorflow主要贡献了CNN,RNN单元的结构定义,损失函数定义,正向反向传播与梯度更新等功能.
\par

\subsection{硬件环境}
本实验几乎所有的运算操作是在一台配置有英伟达GTX1070图形处理器,英特尔i7中央处理器,24GB内存的笔记本电脑上进行的。
\par
本实验深度学习计算部分使用了GPU加速,直接依赖Tensorflow的GPU选项进行。本研究曾尝试过只用CPU进行计算,也能得到一定结果。
\par
如果有更好的硬件条件(更多、更好的图形处理器,更大的内存,更多核心的CPU),本实验有希望会得到更精细的结果。

\section{实验数据}
本实验使用VOT2016数据集\supercite{Vojir-TR-2017-01}实现,相似的数据集还有VOT2017等。
\par
\begin{figure}[htbp!]
    \centering
    \includegraphics[width = 1.\textwidth]{chap/img/vot_2016_pixel.png}
    \caption{VOT2016像素级标记}\label{fig:vot_2016_pixel}
\end{figure}
\par
如图\ref{fig:vot_2016_pixel}所示,该数据集通过人工标记,提供了十分优秀的像素级的目标跟踪数据。该数据集有几百个序列,共有几万张图片。
\par
本实验的训练集和测试集均来源于该数据集,使用时将所有数据随机切分为训练集和测试集。


\section{实验程序的编写}
事实上,虽然借助于Tensorflow实现了许多计算功能,但本研究依然经历了许多代码开发工作,包括且不限于神经网络结构定义,训练数据处理等.

\section{实验结果的评估方式}
	% Copyright (C) 2019 Cui Jialiang ( SESS, PKU )。 All rights reserved。

\chapter{结论与展望}
本章将总结研究的结论,分析并罗列论文的创新点与不足,并对接下来的工作进行展望。

\section{结论}
论文将CNN和RNN融合,实现了一种像素级视频目标跟踪算法。论文利用图像分割的最新算法,使用编码-解码结构的像素级空间处理方法处理空间维度信息。该处理方法使论文提出的算法具有多尺度空间处理效果。在多尺度结构基础上,论文将RNN加入了对应不同尺度的各级编码-解码卷积层,利用RNN对时间序列的强大分析能力连接前后帧图像,实现视频序列处理。
\par
论文提出的算法只使用了一个神经网络处理视频。这种的结构使论文处理过程简单,避免了以往跟踪算法多种算法融合导致的结构复杂、参数冗余大的问题。论文提出的算法只需要进行一个端到端的训练即可训练好神经网络的参数,不需要进行迁移学习,大大降低了学习过程的执行成本。
\par
针对视频中目标具有大致位置、轮廓细节等不同特征以及目标大小不一的情况,论文设计了多尺度的处理结构。该结构使单帧图像处理、视频跟踪状态的保存均具有多尺度性,提升了多尺度跟踪效果。
\par
论文设计实验并验证了论文提出的算法,确定算法具有较好的跟踪能力,并将算法程序开源。针对像素级视频目标跟踪算法缺乏定量评估的情况,论文将成熟应用于推荐系统分类问题的AUC指标引入像素级目标跟踪的评价,实现了定量化的像素级视频目标跟踪评价方式,为像素级视频目标跟踪算法之后的研究与评价奠定了基础。
\par
近年来像素级别的跟踪一直缺乏研究,而像素级的跟踪更加贴近跟踪本质。论文证明了加入了RNN结构的编码-解码模型在像素级目标跟踪问题上的效果,也发掘了像素级别目标跟踪问题的研究空间,有利于推动像素级跟踪算法的研究与迭代。

\section{展望}
本文的后续工作可以在以下方面进行:
\begin{itemize}
    \item 实现目标发现到跟踪的整个过程,尝试进行完整的工业化,并尝试真实应用场景的数据的跟踪效果。
    \item 尝试该结构(RNN加入各级编码-解码结构)在其它问题,如视频目标分割,视频稳定等场景的应用。
\end{itemize}
\par
根据目前的研究趋势,像素级以及矩形级的目标跟踪的推动都主要依靠深度学习算法。近期的深度学习研究主要在修改深度神经网络的模型结构,以求得到更适应问题的神经网络,并得到更好的结果。神经网络结构的改进包括基础网络单元,如CNN,RNN,以及本文的块状CNN,也包括各种基础网络单元结构的组合方式。
\par
对于跟踪问题,论文中认为迫切需要解决的网络结构问题有:
\begin{itemize}
    \item 更好的RNN单元,更能保留有效信息,更好的长效处理。
    \item 更有效的RNN与CNN结合方。
    \item 更多的神经元,容纳更多模型参数,得到更泛化的结果。
\end{itemize}
\par
但在模型结构,训练方法,数据集等问题逐渐完善后,深度学习算法何去何从,如何产生更加优于当前解法的模型将是一个问题。一些学者研究了网络结构的学习\supercite{cortes2017adanet},试图使用计算机替代人类进行网络结构的调整,这样可以使网络结构调整进入'工业化'阶段,大大提高效率。但也有人担心这样容易过早形成强人工智能\supercite{kurzweil2005singularity},对人类生存带来威胁。

% vim:ts=4:sw=4


	% 正文中的附录部分。
	\appendix
	% 排版参考文献列表。bibintoc 选项使“参考文献”出现在目录中;
	% 如果同时要使参考文献列表参与章节编号,可将“bibintoc”改为“bibnumbered”。
	\printbibliography[heading = bibintoc]
	% 各附录。
	%\include{chap/encl1}

	% 以下为正文之后的部分,默认不进行章节编号。
	\backmatter
	% 致谢。
	% Copyright (c) 2014,2016 Casper Ti. Vector
% Public domain.

\chapter{致谢}
这是致谢的第一段
\par
这是致谢的第二段

% vim:ts=4:sw=4

	% 原创性声明和使用授权说明。
	\include{chap/originauth}
\end{document}

% vim:ts=4:sw=4
